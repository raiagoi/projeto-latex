\section{Introdução}
A preservação de tecidos moles em estruturas fossilizadas não é comum devido ao seu complexo processo de fossilização. Para além de estruturas preservadas por completo como em âmbar, no gelo, ou por dessecação, esses tecidos podem ser preservados quando ocorre, por exemplo, soterramento rápido, ausência de oxigênio e a formação de concreções minerais de modo a estabilizar as biomoléculas e inibir sua decomposição completa. Os processos de mineralização autigênica são facilitados pela oferta de elementos químicos para formar minerais, assim como pela vida microbiana local; assim, enquanto a estrutura é preservada, a composição se altera \cite{brigs2003}. 

%teste de alteração
A formação da pirita como resultado da diagênese (processos pós o soterramento de uma carcaça) é denominada piritização. Para isto, há a necessidade de alta concentração em sulfato, em média 28 mM \cite{Farrell_2014}, sedimentos de granulação fina, com baixas concentrações de carbono orgânico, e com ferro ambiental reativo para a formação da pirita na carcaça em decomposição \cite{Farrell_2014}, por bactérias redutoras de sulfato.


Conforme mencionado anteriormente, não é usual a piritização de tecidos moles. No entanto, esse processo foi observado em músculos e olhos de vertebrados (peixes) fossilizados na Bacia Sedimentar do Araripe (Ceará; Período Cretáceo, ca. 110 milhões de anos atrás) \cite{oses2017}. A identificação baseou-se na constatação de que esses fósseis são predominantemente compostos por óxidos e hidróxidos de ferro, produtos da oxidação da pirita. Entre as unidades geológicas que compõem o preenchimento da Bacia do Araripe, a Formação Crato se destaca por ser constituída por calcários intercalados com folhelhos, arenitos e siltitos. Essa formação representa o registro paleobiológico continental mais completo do Cretáceo, sendo mundialmente reconhecida pela grande diversidade e pelo excepcional estado de preservação de seus fósseis. Ela oferece uma janela única para a obtenção de informações morfológicas e ecológicas raramente preservadas no registro fóssil.

Além de insetos, peixes e plantas, também são encontrados na Formação Crato. Diversos estudos têm se dedicado à análise da preservação de detalhes morfológicos externos e de tecidos internos em insetos da Formação Crato \cite{Barling2015, Oses2016, DiasCarvalho2020, STORARI2024}.


Para além dos óxidos e hidróxidos de ferro, também foi identificada a formação de fosfato decorrente do processo de fosfatização \cite{Oses2016}. Esse processo também implica em uma mineralização da matéria orgânica, que pode ocorrer pela liberação de fosfato no ambiente a partir de fitoplâncton e sendo facilitada pela atividade microbiana \cite{Schiffbauer_Wallace_Broce_Xiao_2014, konhause}.

A fosfatização geralmente se inicia pela decomposição inicial de material orgânico, acumulação de fosfato em estruturas orgânicas e, em seguida, mineralização, que pode preservar detalhes celulares finos. Os complexos processos geoquímicos incluem a nucleação e o crescimento de minerais de fosfato em substratos biológicos, impulsionados pela degradação de materiais orgânicos, que enriquece os fluidos circundantes com fosfato \cite{Schiffbauer_Wallace_Broce_Xiao_2014}. 

Análises sobre os tecidos moles fossilizados vêm inovando o conhecimento sobre a biologia de organismos extintos, abrindo possibilidades para compreender de forma mais completa o que as partes duras dos organismos não possibilitam informar, para além de contribuir para investigar os mecanismos responsáveis por sua preservação. De forma não destrutiva, foram desenvolvidas técnicas de imageamento por raio X, possibilitando a análise de estruturas que ficam parcial ou totalmente ocultas na observação direta, dispensando a remoção mecânica ou química da rocha que os recobre, permitindo a visualização de porções internas dos espécimes; contudo, isso se limita a técnica de imagem, sendo impossível identificar dados composicionais. Estas técnicas consistem em uma grande ferramentas de aplicabilidade na caracterização de fósseis (i.e., Paleometria – \cite{Riquelme2009, Delgado2014, Gomes2019}, 

\cite{Prado2021}).


Para a finalidade de identificação composicional da parte interior das amostras, até então foram utilizadas técnicas não destrutivas associadas a destrutivas, como corte e polimento, infligindo danos irreversíveis às amostras. Este estudo apresenta uma forma inovadora de realizar análises químicas do interior de fósseis de forma totalmente não destrutiva, que envolve a utilização do PIXE diferencial para uma análise multi-elementos em baixas concentrações.


O PIXE diferencial consiste na incidência de um feixe de prótons no material analisado, de modo a induzir a emissão de raios X por ionização (PIXE), a depender do material e da energia aplicada, assim permitindo medir-se a composição interna sem a necessidade de corte, exposição e medida da amostra. Com prótons com alta energia incidentes em átomos usualmente estáveis, haverá uma mudança de estado de elétron e ionização, fazendo com que elétrons em estados mais energéticos ocupem a vacância gerada pelos ejetados, liberando energia caracteristicas dos elementos químicos na forma de ondas eletromagnéticas na faixa do raio X \cite{ishii2}. Este processo se dá de forma distinta em cada elemento, podendo ser utilizado então para identificar cada material, a depender da energia característica liberada.

As análises com PIXE diferencial possuem limitações e dificuldades, tendo como exemplo um limite delimitado pelos raios-X emitidos passarem através dos detectores sem serem reconhecidos \cite{ISHII1988209}, para além de suas incertezas devido a perdas de absorção das camadas superiores se não bem estimadas \cite{SMIT20082047}. Para além disso, há dificuldades nas próprias amostras investigadas, pois os estudos visando a caracterização da composição química de diferentes tecidos mineralizados em perfil de profundidade nas amostras ainda estão sendo realizados, não possuindo, ainda, dados comparativos, o que, por outro lado, mostra sua inovação nesta área. Essa demanda também está aberta para a caracterização do patrimônio paleontológico em geral. 


A caracterização por PIXE permite sondar amostras para avaliar a presença de diferentes tecidos e investigar a composição dessas estruturas e de seus processos de fossilização. Apesar de não haver muitas pesquisas com PIXE diferencial em materiais fossilizados, resultados anteriores já foram úteis para a caracterização da composição de diferentes amostras. Estudos com PIXE de fósseis e conchas recentes da Baja Califórnia, México \cite{oliver1996chemical} foram úteis para determinar concentrações de metais, dentre eles alguns com quantidades muito maiores em comparação com conchas recentes. Outro estudo demonstrou o sucesso para analisar a composição química em tecidos moles preservados e ossos, evidenciando composições minerais para cada ponto estudado para os fósseis da Formação Tlayúa, México \cite{Riquelme2009}. Por último, pode-se notar que, também para o estudo de tecidos moles mineralizados, o PIXE permitiu a descoberta de uma alta concentração de Mn na rocha matriz dos fósseis, indicando uma disseminação de pirolusita, além da oxidação e/ou a hidratação da pirita formada no processo de fossilização que geraram óxidos ou hidróxidos de ferro \cite{Oses2016}.


Assim, a análise não destrutiva do interior de amostras fósseis é de extrema importância para preservar a integridade de espécimes raros e valiosos, permitindo estudos repetidos e futuras investigações sem danos irreversíveis. Além disso, essa abordagem possibilita a caracterização química em diferentes profundidades, essencial para entender processos de fossilização, por exemplo, a piritização. A presente pesquisa tem como objetivo desenvolver e aplicar a técnica de PIXE diferencial para sondar a composição interna de fósseis da Bacia do Araripe de forma totalmente não destrutiva, superando as limitações das metodologias tradicionais que exigem corte ou polimento das amostras. Essa inovação não apenas preserva o patrimônio paleontológico, mas também abre novas perspectivas para a compreensão da biologia de organismos extintos e dos ambientes em que viveram em consonância com sua preservação. 
