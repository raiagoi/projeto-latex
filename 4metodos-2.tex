\subsection{Métodos de Medição}

As amostras foram analisadas por PIXE no Laboratório de Análise de Materiais por Feixes Iônicos (LAMFI/USP). As medições foram realizadas utilizando um feixe de prótons com energia de \SI{2.6}{\mega\electronvolt} a \SI{3.4}{\mega\electronvolt} e corrente de \SI{3}{\nano\ampere} a \SI{5}{\nano\ampere}. O diâmetro do feixe foi ajustado para aproximadamente \SI{1}{\milli\metre}, e o tempo de irradiação de cada ponto foi de cerca de \SI{600}{\second}, garantindo boa estatística sem degradação perceptível das amostras.

O arranjo experimental se baseia no feixe externo, como se mostra como na figura \ref{fig:medicao_pixie_dupla}, onde a amostra esta posicionada a frente do feixe e do detector.
\begin{figure}[H]
	\centering
	
	\includegraphics[width=0.7\textwidth]{Métodos/Captura de tela de 2025-10-24 19-48-51-1.png}\\[1em]
	\includegraphics[width=0.7\textwidth]{Métodos/Captura de tela de 2025-10-24 19-49-27-1.png}
	\caption{Vistas do sistema experimental durante as medições PIXE realizadas no LAMFI: superior exibe uma visão geral do arranjo experimental; inferior exibe amostra de frente.}
	\label{fig:medicao_pixie_dupla}
\end{figure}


Os espectros obtidos foram posteriormente processados e calibrados em energia e intensidade, pelo software QXAS (\textit{Quantitative X-ray Analysis System}), permitindo a determinação quantitativa dos elementos presentes nas amostras. 


\subsection{Métodos de Análise}
\subsection{Aproximação para alvos finos}
Inicialmente para estimar a espessura dos padrões produzidos, foi aproximado sua superfície a filmes finos, de modo a possibilitar o cálculo a partir da equação \ref{eq:alvosfinos}, porém algumas mudanças precisam ser feitas para ter uma melhor aproximação neste caso. Inicialmente há de se considerar que a composição da camada superior tem uma presença de oxigênio, que não é detectado devido aos parâmetros do PIXE. Utilizando da proporção de massa molar de Ferro presente na molécula de Fe${}_2$O${}_3$, temos uma conversão da concentração total em função da concentração de Ferro dada por

\begin{equation}
	C_{total} \approx C_{Fe}  \cdot 1,43
	\label{eq:concentracao_total}
\end{equation}
em que pode ser estimada sem a necessidade de medidas diretas do oxigênio contido na amostra.

Para encontrar a espessura é preciso correlacionar com a densidade superficial característica do óxido de ferro em questão, sendo em $1 \micro$m tendo $523,96\micro$m $cm^{-2}$. Assim convertendo a equação \ref{eq:alvosfinos} com estes novos parâmetro para encontrar as espessuras média tem-se

\begin{equation}
	d_{Fe_2O_3} = \frac{Y^{Fe}_{K_\alpha}}{Q_{Fe_2O_3}  \cdot r_{K_\alpha}^{E_0}}  \cdot 2,73  \cdot 10^{-3}
	\label{eq:d_alvos_finos}
\end{equation}

com o resultado sendo obtido em $\micro$m. A limitação experimental que se obtêm é somente para a obtenção da corrente de $Q_{Fe_2O_3}$ isoladamente, pois somente é obtida a corrente total da amostra. Porém para uma primeira aproximação é suficiente para investigar estes casos, além de servir como base para a expansão para alvos espessos.


\begin{equation}
	d_{Fe_2O_3} = \frac{Y^{Fe}_{K_\alpha}}{Q \cdot r_{K_\alpha}^{E_0}}  \cdot 2,73  \cdot 10^{-3}
	\label{eq:d_alvos_finos}
\end{equation}
  
\subsection{Aproximação para alvos espesso}

Como desenvolvido seguido as equações \ref{eq:alvosfinos}, \ref{eq:Alvos_grossos_inic} e \ref{eq:Alvos_grossosf_fin}, temos que $\tilde{Y}^{Fe}_{K_\alpha} = Y^{Fe}_{K_\alpha} \cdot R_M $ sendo a contagem para a extensão em alvos espessos $\tilde{Y}^{Fe}_{K_\alpha}$, e 
$Y^{Fe}_{K_\alpha}$ a aproximação inicial de alvos finos. Calculando o valor de $R_M$, é necessário utilizar de aproximações e métodos já consolidados sobre estas variáveis. Inicialmente para a seção de choque, foi selecionado o ECPSSR, um método analítico que estende a aproximação de Born de onda plana (PWBA), que considera a perda de energia (E), a deflexão de Coulomb (C) do projétil, o estado de pertubação estacionária (PSS) e relativística (R), desenvolvido em código como \cite{LiuCipolla1996ISICS}. Para o poder de freamento, é utilizado dos valores tabelados por SRIM \cite{ZIEGLER20101818}{\color{red} TALVEZ SEJA MELHOR AJUSTAR UM POLINÔMIO PARA OS DADOS E UTILIZAR OS VALORES DESSA CURVA}. Já para o coeficiente de atenuação de massa, é utilizado do valor tabelado pela base de dados do FFAST \cite{HubbellChantlerNISTtables}. O feixe de incidência é perpendicular a superfície da amostra, e o do detector se encontra em 45º \cite{RIZZUTTO2005549}. 

\subsection{PIXE diferencial}
 A solução envolve minimizar $\chi^2$ dado por 
\begin{equation}
	\chi^{2} = \sum_{i k} \left( \frac{Y_{i}^{k*}}{A_{k}} - \sum_{j} T_{ij}^{k} x_{ij} \right)^{2}.
\end{equation}
até encontrar valores satisfatórios para $x_{ij}$




