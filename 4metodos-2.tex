\subsection{Métodos de Medição}

As amostras foram analisadas por PIXE no Laboratório de Análise de Materiais por Feixes Iônicos (LAMFI/USP). As medições foram realizadas utilizando um feixe de prótons com energia de \SI{2.6}{\mega\electronvolt} a \SI{3.4}{\mega\electronvolt} e corrente de \SI{3}{\nano\ampere} a \SI{5}{\nano\ampere}. O diâmetro do feixe foi ajustado para aproximadamente \SI{1}{\milli\metre}, e o tempo de irradiação de cada ponto foi de cerca de \SI{600}{\second}, garantindo boa estatística sem degradação perceptível das amostras.

O arranjo experimental se baseia no feixe externo, como se mostra como na figura \ref{fig:medicao_pixie_dupla}, onde a amostra esta posicionada a frente do feixe e do detector.
\begin{figure}[H]
	\centering
	
	\includegraphics[width=0.7\textwidth]{Métodos/Captura de tela de 2025-10-24 19-48-51-1.png}\\[1em]
	\includegraphics[width=0.7\textwidth]{Métodos/Captura de tela de 2025-10-24 19-49-27-1.png}
	\caption{Vistas do sistema experimental durante as medições PIXE realizadas no LAMFI: superior exibe uma visão geral do arranjo experimental; inferior exibe amostra de frente.}
	\label{fig:medicao_pixie_dupla}
\end{figure}


Os espectros obtidos foram posteriormente processados e calibrados em energia e intensidade, pelo software QXAS (\textit{Quantitative X-ray Analysis System}), permitindo a determinação quantitativa dos elementos presentes nas amostras. 


\subsection{Métodos de Análise}

 A solução envolve minimizar $\chi^2$ dado por 
\begin{equation}
	\chi^{2} = \sum_{i k} \left( \frac{Y_{i}^{k*}}{A_{k}} - \sum_{j} T_{ij}^{k} x_{ij} \right)^{2}.
\end{equation}
até encontrar valores satisfatórios para $x_{ij}$




