\section{Resultados}
\subsection{Alvos finos preliminares}

\begin{table}[H]
	\centering
	\caption{Valores de contagem de $Y^{Fe}_{K_\alpha}$ com incerteza, carga acumulada e espessura calculada para cada amostra.}
	\label{tab:dados_fe_incerteza}
	\begin{tabular}{c c c c}
		\hline
		Amostra & $Y^{Fe}_{K_\alpha}$(Contagens) & $Q$($\mu$C) & $d$($\mu$m) \\
		\hline
		1  & $1751(13)\times 10^{1}$ & 0.555 & 0.513259 \\
		2  & $841(92)\times 10^{1}$  & 0.534 & 0.256102 \\
		3  & $2381(16)\times 10^{1}$ & 0.484 & 0.800411 \\
		4  & $1140(11)\times 10^{1}$ & 0.494 & 0.375581 \\
		5  & $927(97)\times 10^{1}$  & 0.508 & 0.296902 \\
		6  & $2519(16)\times 10^{1}$ & 0.471 & 0.870076 \\
		7  & $2529(16)\times 10^{1}$ & 0.448 & 0.918701 \\
		8  & $4301(21)\times 10^{1}$ & 0.482 & 1.4519   \\
		9  & $2498(16)\times 10^{1}$ & 0.436 & 0.932073 \\
		10 & $5601(24)\times 10^{1}$ & 0.405 & 2.25031  \\
		\hline
	\end{tabular}
\end{table}



{\color{red} CALCULAR INCERTEZA, CONSIDERAR A DISTRIBUIÇÃO DE CARGA, E DO CALCULO DE SECCHOQUE}

\begin{figure}[H]
	\centering
	\begin{subfigure}[b]{0.495\linewidth}
	\includegraphics[width=\linewidth]{results/grafico 2 com pico2.png}
	\label{fig:graf2-fit}
	\end{subfigure}
	\hfill
	\begin{subfigure}[b]{0.495\linewidth}
	\includegraphics[width=\linewidth]{results/grafico 8 com pico.png}
	\label{fig:graf8-fit}
	\end{subfigure}
	\hfill
	\caption{Gráfico de contagens pelo canal com ajuste para a pastilha 2 (a) com $\chi^2 = 1.9$, e 8 (b) com $\chi^2 = 1.4$}
	
\end{figure}

\

\subsection{Pixe diferencial}
Foram realizadas medidas de PIXE com três valores de energia diferentes, nas pastilhas duplas e adsorvidas, para criar um padrão comparativo quando os filmes finos estivessem prontos. Com isso, as energias dos prótons incidentes foram de 2.6, 3.0 e 3.4 MeV com um tempo de 300 segundos de medição (Figs. \ref{figbipellet} e \ref{fig:ads} ).
\begin{figure}[H]
	\centering
	\includegraphics[width=\linewidth]{results/BLP-1-1.png}
	\caption{Gráfico de contagens pelo canal, das análises da pastilha duplas}
	\label{figbipellet}
\end{figure}
Observando os resultados obtidos para a pastilha dupla, é notável que a espessura superior da pastilha foi grossa o suficiente para impedir que os Raios-X emitidos transpassassem por ela, sendo absorvidos no trajeto. Foi possível identificar o Fe presente e uma baixa concentração de Ca que não variou com o aumento na energia.

\begin{figure}[H]
	\centering
	\includegraphics[width=\linewidth]{results/ADS-1-1.png}
	\caption{Gráfico de contagens pelo canal, das análises da pastilha adsorvidas}
	\label{fig:ads}
\end{figure}
Para esta última, os valores de $\chi^2$ para a pastilha dupla e adsorvida foram aproximadamente 5 e 3, respectivamente. Sendo assim, é notável a variação das contagens de Ca e P em função da energia, representando que com uma variação da energia, a variação das contagens mudou significantemente, mesmo normalizando pela corrente. Essa variação pode ser apenas referente ao raio X emitido pela energia, ou pode indicar que o feixe esta lendo camadas distintas da amostra. Esta questão será sanada no próximo passo, com a aplicação do algorítimo de deconvolução. 
