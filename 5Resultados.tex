\section{Resultados e discussão }
Foram realizadas medidas de PIXE com três valores de energia diferentes, nas pastilhas duplas e adsorvidas, para criar um padrão comparativo quando os filmes finos estivessem prontos. Com isso, as energias dos prótons incidentes foram de 2.6, 3.0 e 3.4 MeV com um tempo de 300 segundos de medição (Figs. \ref{figbipellet} e \ref{fig:ads} ).
\begin{figure}[H]
	\centering
	\includegraphics[width=\linewidth]{results/BLP-1-1.png}
	\caption{Gráfico de contagens pelo canal, das análises da pastilha duplas}
	\label{figbipellet}
\end{figure}
Observando os resultados obtidos para a pastilha dupla, é notável que a espessura superior da pastilha foi grossa o suficiente para impedir que os Raios-X emitidos transpassassem por ela, sendo absorvidos no trajeto. Foi possível identificar o Fe presente e uma baixa concentração de Ca que não variou com o aumento na energia.

\begin{figure}[H]
	\centering
	\includegraphics[width=\linewidth]{results/ADS-1-1.png}
	\caption{Gráfico de contagens pelo canal, das análises da pastilha adsorvidas}
	\label{fig:ads}
\end{figure}
Para esta última, os valores de $\chi^2$ para a pastilha dupla e adsorvida foram aproximadamente 5 e 3, respectivamente. Sendo assim, é notável a variação das contagens de Ca e P em função da energia, representando que com uma variação da energia, a variação das contagens mudou significantemente, mesmo normalizando pela corrente. Essa variação pode ser apenas referente ao raio X emitido pela energia, ou pode indicar que o feixe esta lendo camadas distintas da amostra. Esta questão será sanada no próximo passo, com a aplicação do algorítimo de deconvolução. 
