\begin{titlepage}
    \begin{center}
        \begin{figure}[!ht]
        \centering
        \includegraphics[width=3.7cm]{imagens/USP-Brasão.jpg}
        \end{figure}
        {\large Relatório parcial de iniciação científica} \\
        \vspace{4mm}
    
        \hrule
        \vspace{4mm}
        \normalsize{\textbf{Utilização de PIXE diferencial para sondar a composição interna fósseis}}\\
        \vspace{4mm}
        \hrule
    
        \vspace{2mm}
        \begin{flushright}
        {
        Período: maio de 2025 à dezembro de 2025}
        \\
        
        \vspace{2mm}
        {
        Orientador: Gabriel Ladeira Osés}   
        \\
        \vspace{2mm}
    
        {
        Orientando: Iago Rojahn da Silva}
        \\
        \vspace{2mm}
        {
        N° USP: 12557417}
    
        \end{flushright}
        
        \begin{abstract}
            Fósseis com preservação excepcional incluem espécimes tridimensionais e com tecidos originais mineralizados. Consequentemente, esses fósseis permitem a reconstituição de relações filogenéticas (parentesco entre organismos), paleoecológicas e das condições paleoambientais, possuindo assim grande interesse de diferentes áreas de pesquisa. A caracterização não destrutiva (ou minimamente invasiva) da composição química de tecidos mineralizados em diferentes profundidades nas amostras é fundamental para avaliar a presença de diferentes tecidos e para a investigação da composição original das estruturas e de seus processos de fossilização. Porém, essa tarefa está em aberto, representando desafio analítico. A Formação Crato (Bacia Sedimentar do Araripe, Ceará) é uma das mais notáveis janelas para a vida no passado do planeta, consistindo no mais completo registro de fósseis de paleoambiente continental do Período Cretáceo (ca. 110 milhões de anos atrás). Os insetos, peixes e plantas fósseis dessa formação possuem detalhes da morfologia externa e interna preservada. Esta pesquisa tem como objetivo principal desenvolver técnica envolvendo PIXE (\textit{Particle Induced X-ray Emission}) diferencial para caracterizar a composição química de diferentes tecidos em diferentes profundidades em fósseis, utilizando amostras da Fm. Crato como modelo. Para atingir esse objetivo, inicialmente, padrões desenvolvidos em laboratório simulando contrastes de composição de estruturas com diferentes espessuras análogos aos fósseis, serão caracterizados por PIXE diferencial. Simulações serão realizadas para determinar-se valores de energias específicos para caracterizar fósseis com perfil interno de variação de composição.
        \end{abstract}
        
        \vspace*{\fill}
        \large{Instituto de Física\\
            São Paulo\\
            2025}
        
    \end{center}
\end{titlepage}


    