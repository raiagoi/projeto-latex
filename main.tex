\documentclass[12pt,a4paper,english,brazil]{exam}
\usepackage[brazilian]{babel}
\usepackage{lmodern}            % Usa a fonte Latin Modern
\usepackage[T1]{fontenc}        % Selecao de codigos de fonte.
\usepackage[utf8]{inputenc}     % Codificacao do documento (conversão automática dos acentos) seção.
\usepackage{breakcites}
\usepackage{setspace}           % Espaçamento
\usepackage{graphicx}      % Para incluir imagens
\usepackage{booktabs}
\usepackage{amsmath}
\usepackage{systeme}
\usepackage{hyperref}  %Referencia no clique
\usepackage{amssymb}
\usepackage{cite}
\usepackage{geometry}
\usepackage{braket}
\geometry{
	a4paper,
	total={170mm,257mm},
	left=20mm,
	top=20mm,
}
\usepackage[version=4]{mhchem}
\usepackage{multicol}
\usepackage{mathtools}
\usepackage{wrapfig}
\usepackage{siunitx}           % Para unidades (SI)
\usepackage{caption}           % Para legendas personalizadas
\usepackage{chemformula}       % Para fórmulas químicas (\ce{})
\usepackage{tikz}
\usepackage{adjustbox}
\usepackage{indentfirst}
\usepackage{float}
\usepackage{ragged2e}
\usepackage{blindtext}
\usepackage{verbatim}
\usepackage{xcolor}
\usepackage{tcolorbox}
\usepackage{gensymb}
\usepackage{pdfpages}
\usepackage{listings} % Adiciona o pacote listings
\usepackage[acronym,style=indexgroup]{glossaries-extra}
\usepackage{ragged2e}
\usepackage[normalem]{ulem}
\usepackage{subcaption}



\usepackage[autostyle]{csquotes}
\MakeOuterQuote{"}

\definecolor{darkpink}{RGB}{255, 20, 147} % Rosa forte (Deep Pink)

\lstset{ % Configurações do pacote listings
	language=C++, % Nome fictício para a linguagem
	morekeywords={run, wait, getPixel, for, if, else, set, exit}, % Palavras-chave do ImageJ Macro
	sensitive=true, % Define a sensibilidade a maiúsculas e minúsculas
	morestring=[b]", % Strings delimitadas por aspas
	morecomment=[l]//, % Comentários de linha
	morecomment=[s]{/*}{*/}, % Comentários de bloco
	basicstyle=\ttfamily\footnotesize, % Define a fonte básica
	keywordstyle=\color{blue}, % Define a cor das palavras-chave
	stringstyle=\color{darkgreen}, % Define a cor das strings
	commentstyle=\color{darkpink}, % Define a cor dos comentários
	numbers=left, % Adiciona numeração das linhas
	numberstyle=\tiny\color{gray}, % Define o estilo da numeração das linhas
	stepnumber=1, % Numera todas as linhas
	numbersep=10pt, % Espaço entre a numeração e o código
	backgroundcolor=\color{white}, % Cor de fundo
	showspaces=false, % Não mostra espaços em branco
	showstringspaces=false, % Não mostra espaços em strings
	showtabs=false, % Não mostra tabulações
	tabsize=4, % Define o tamanho da tabulação
	captionpos=b, % Posição da legenda
	breaklines=true, % Permite quebra de linha
	breakatwhitespace=false, % Quebra de linha em espaços em branco
	escapeinside={\%*}{*)} % Para incluir comandos LaTeX dentro do código
}

\lstnewenvironment{cpp}[1][]{%
	\lstset{
		language=C++, % Escolha a linguagem de programação
		basicstyle=\ttfamily\footnotesize, % Define a fonte básica
		keywordstyle=\color{blue}, % Define a cor das palavras-chave
		stringstyle=\color{darkgreen}, % Define a cor das strings
		commentstyle=\color{darkpink}, % Define a cor dos comentários
		numbers=left, % Adiciona numeração das linhas
		numberstyle=\tiny\color{gray}, % Define o estilo da numeração das linhas
		stepnumber=1, % Numera todas as linhas
		numbersep=10pt, % Espaço entre a numeração e o código
		backgroundcolor=\color{white}, % Cor de fundo
		showspaces=false, % Não mostra espaços em branco
		showstringspaces=false, % Não mostra espaços em strings
		showtabs=false, % Não mostra tabulações
		tabsize=4, % Define o tamanho da tabulação
		captionpos=b, % Posição da legenda
		breaklines=true, % Permite quebra de linha
		escapeinside={\%*}{*)} % Para incluir comandos LaTeX dentro do código
}}{}

\makeatletter
\renewcommand\tableofcontents{%         % Redefine table of contents to our taste
	\section*{\huge\centering\contentsname
		\@mkboth{%
			\MakeUppercase\contentsname}{\MakeUppercase\contentsname}}%
	\vspace{24pt}%
	\@starttoc{toc}%
	\thispagestyle{empty}
}


\usetikzlibrary{calc,patterns,angles,quotes}


\definecolor{darkgreen}{RGB}{0, 100, 0}
\captionsetup[subfigure]{labelformat=empty, font=footnotesize}


\renewcommand*\contentsname{Sumário}
\renewcommand{\figurename}{\textbf{Figura}}
\renewcommand{\tablename}{\textbf{Tabela}}



\newcommand{\sg}{$\sigma$}
\newcommand{\omg}{$\Omega$}
\newcommand{\gr}{\degree}
\newcommand{\gm}{$\gamma$}
\newcommand{\dl}{$\delta$}
\newcommand{\m}{$\mu$}
\newcommand{\mgcm}{\m $g$/$cm^2$}
\begin{document}
	\begin{titlepage}
    \begin{center}
        \begin{figure}[!ht]
        \centering
        \includegraphics[width=3.7cm]{imagens/USP-Brasão.jpg}
        \end{figure}
        {\large Relatório parcial de iniciação científica} \\
        \vspace{4mm}
    
        \hrule
        \vspace{4mm}
        \normalsize{\textbf{Utilização de PIXE diferencial para sondar a composição interna fósseis}}\\
        \vspace{4mm}
        \hrule
    
        \vspace{2mm}
        \begin{flushright}
        {
        Período: maio de 2025 à dezembro de 2025}
        \\
        
        \vspace{2mm}
        {
        Orientador: Gabriel Ladeira Osés}   
        \\
        \vspace{2mm}
    
        {
        Orientando: Iago Rojahn da Silva}
        \\
        \vspace{2mm}
        {
        N° USP: 12557417}
    
        \end{flushright}
        
        \begin{abstract}
            Fósseis com preservação excepcional incluem espécimes tridimensionais e com tecidos originais mineralizados. Consequentemente, esses fósseis permitem a reconstituição de relações filogenéticas (parentesco entre organismos), paleoecológicas e das condições paleoambientais, possuindo assim grande interesse de diferentes áreas de pesquisa. A caracterização não destrutiva (ou minimamente invasiva) da composição química de tecidos mineralizados em diferentes profundidades nas amostras é fundamental para avaliar a presença de diferentes tecidos e para a investigação da composição original das estruturas e de seus processos de fossilização. Porém, essa tarefa está em aberto, representando desafio analítico. A Formação Crato (Bacia Sedimentar do Araripe, Ceará) é uma das mais notáveis janelas para a vida no passado do planeta, consistindo no mais completo registro de fósseis de paleoambiente continental do Período Cretáceo (ca. 110 milhões de anos atrás). Os insetos, peixes e plantas fósseis dessa formação possuem detalhes da morfologia externa e interna preservada. Esta pesquisa tem como objetivo principal desenvolver técnica envolvendo PIXE (\textit{Particle Induced X-ray Emission}) diferencial para caracterizar a composição química de diferentes tecidos em diferentes profundidades em fósseis, utilizando amostras da Fm. Crato como modelo. Para atingir esse objetivo, inicialmente, padrões desenvolvidos em laboratório simulando contrastes de composição de estruturas com diferentes espessuras análogos aos fósseis, serão caracterizados por PIXE diferencial. Simulações serão realizadas para determinar-se valores de energias específicos para caracterizar fósseis com perfil interno de variação de composição.
        \end{abstract}
        
        \vspace*{\fill}
        \large{Instituto de Física\\
            São Paulo\\
            2025}
        
    \end{center}
\end{titlepage}


    
	
	\tableofcontents
	\newpage
	
	\section{Introdução}
A preservação de tecidos moles em estruturas fossilizadas não é comum devido ao seu complexo processo de fossilização. Para além de estruturas preservadas por completo como em âmbar, no gelo, ou por dessecação, esses tecidos podem ser preservados quando ocorre, por exemplo, soterramento rápido, ausência de oxigênio e a formação de concreções minerais de modo a estabilizar as biomoléculas e inibir sua decomposição completa. Os processos de mineralização autigênica são facilitados pela oferta de elementos químicos para formar minerais, assim como pela vida microbiana local; assim, enquanto a estrutura é preservada, a composição se altera \cite{brigs2003}. 

%teste de alteração
A formação da pirita como resultado da diagênese (processos pós o soterramento de uma carcaça) é denominada piritização. Para isto, há a necessidade de alta concentração em sulfato, em média 28 mM \cite{Farrell_2014}, sedimentos de granulação fina, com baixas concentrações de carbono orgânico, e com ferro ambiental reativo para a formação da pirita na carcaça em decomposição \cite{Farrell_2014}, por bactérias redutoras de sulfato.


Conforme mencionado anteriormente, não é usual a piritização de tecidos moles. No entanto, esse processo foi observado em músculos e olhos de vertebrados (peixes) fossilizados na Bacia Sedimentar do Araripe (Ceará; Período Cretáceo, ca. 110 milhões de anos atrás) \cite{oses2017}. A identificação baseou-se na constatação de que esses fósseis são predominantemente compostos por óxidos e hidróxidos de ferro, produtos da oxidação da pirita. Entre as unidades geológicas que compõem o preenchimento da Bacia do Araripe, a Formação Crato se destaca por ser constituída por calcários intercalados com folhelhos, arenitos e siltitos. Essa formação representa o registro paleobiológico continental mais completo do Cretáceo, sendo mundialmente reconhecida pela grande diversidade e pelo excepcional estado de preservação de seus fósseis. Ela oferece uma janela única para a obtenção de informações morfológicas e ecológicas raramente preservadas no registro fóssil.

Além de insetos, peixes e plantas, também são encontrados na Formação Crato. Diversos estudos têm se dedicado à análise da preservação de detalhes morfológicos externos e de tecidos internos em insetos da Formação Crato \cite{Barling2015, Oses2016, DiasCarvalho2020, STORARI2024}.


Para além dos óxidos e hidróxidos de ferro, também foi identificada a formação de fosfato decorrente do processo de fosfatização \cite{Oses2016}. Esse processo também implica em uma mineralização da matéria orgânica, que pode ocorrer pela liberação de fosfato no ambiente a partir de fitoplâncton e sendo facilitada pela atividade microbiana \cite{Schiffbauer_Wallace_Broce_Xiao_2014, konhause}.

A fosfatização geralmente se inicia pela decomposição inicial de material orgânico, acumulação de fosfato em estruturas orgânicas e, em seguida, mineralização, que pode preservar detalhes celulares finos. Os complexos processos geoquímicos incluem a nucleação e o crescimento de minerais de fosfato em substratos biológicos, impulsionados pela degradação de materiais orgânicos, que enriquece os fluidos circundantes com fosfato \cite{Schiffbauer_Wallace_Broce_Xiao_2014}. 

Análises sobre os tecidos moles fossilizados vêm inovando o conhecimento sobre a biologia de organismos extintos, abrindo possibilidades para compreender de forma mais completa o que as partes duras dos organismos não possibilitam informar, para além de contribuir para investigar os mecanismos responsáveis por sua preservação. De forma não destrutiva, foram desenvolvidas técnicas de imageamento por raio X, possibilitando a análise de estruturas que ficam parcial ou totalmente ocultas na observação direta, dispensando a remoção mecânica ou química da rocha que os recobre, permitindo a visualização de porções internas dos espécimes; contudo, isso se limita a técnica de imagem, sendo impossível identificar dados composicionais. Estas técnicas consistem em uma grande ferramentas de aplicabilidade na caracterização de fósseis (i.e., Paleometria – \cite{Riquelme2009, Delgado2014, Gomes2019}, 

\cite{Prado2021}).


Para a finalidade de identificação composicional da parte interior das amostras, até então foram utilizadas técnicas não destrutivas associadas a destrutivas, como corte e polimento, infligindo danos irreversíveis às amostras. Este estudo apresenta uma forma inovadora de realizar análises químicas do interior de fósseis de forma totalmente não destrutiva, que envolve a utilização do PIXE diferencial para uma análise multi-elementos em baixas concentrações.


O PIXE diferencial consiste na incidência de um feixe de prótons no material analisado, de modo a induzir a emissão de raios X por ionização (PIXE), a depender do material e da energia aplicada, assim permitindo medir-se a composição interna sem a necessidade de corte, exposição e medida da amostra. Com prótons com alta energia incidentes em átomos usualmente estáveis, haverá uma mudança de estado de elétron e ionização, fazendo com que elétrons em estados mais energéticos ocupem a vacância gerada pelos ejetados, liberando energia caracteristicas dos elementos químicos na forma de ondas eletromagnéticas na faixa do raio X \cite{ishii2}. Este processo se dá de forma distinta em cada elemento, podendo ser utilizado então para identificar cada material, a depender da energia característica liberada.

As análises com PIXE diferencial possuem limitações e dificuldades, tendo como exemplo um limite delimitado pelos raios-X emitidos passarem através dos detectores sem serem reconhecidos \cite{ISHII1988209}, para além de suas incertezas devido a perdas de absorção das camadas superiores se não bem estimadas \cite{SMIT20082047}. Para além disso, há dificuldades nas próprias amostras investigadas, pois os estudos visando a caracterização da composição química de diferentes tecidos mineralizados em perfil de profundidade nas amostras ainda estão sendo realizados, não possuindo, ainda, dados comparativos, o que, por outro lado, mostra sua inovação nesta área. Essa demanda também está aberta para a caracterização do patrimônio paleontológico em geral. 


A caracterização por PIXE permite sondar amostras para avaliar a presença de diferentes tecidos e investigar a composição dessas estruturas e de seus processos de fossilização. Apesar de não haver muitas pesquisas com PIXE diferencial em materiais fossilizados, resultados anteriores já foram úteis para a caracterização da composição de diferentes amostras. Estudos com PIXE de fósseis e conchas recentes da Baja Califórnia, México \cite{oliver1996chemical} foram úteis para determinar concentrações de metais, dentre eles alguns com quantidades muito maiores em comparação com conchas recentes. Outro estudo demonstrou o sucesso para analisar a composição química em tecidos moles preservados e ossos, evidenciando composições minerais para cada ponto estudado para os fósseis da Formação Tlayúa, México \cite{Riquelme2009}. Por último, pode-se notar que, também para o estudo de tecidos moles mineralizados, o PIXE permitiu a descoberta de uma alta concentração de Mn na rocha matriz dos fósseis, indicando uma disseminação de pirolusita, além da oxidação e/ou a hidratação da pirita formada no processo de fossilização que geraram óxidos ou hidróxidos de ferro \cite{Oses2016}.


Assim, a análise não destrutiva do interior de amostras fósseis é de extrema importância para preservar a integridade de espécimes raros e valiosos, permitindo estudos repetidos e futuras investigações sem danos irreversíveis. Além disso, essa abordagem possibilita a caracterização química em diferentes profundidades, essencial para entender processos de fossilização, por exemplo, a piritização. A presente pesquisa tem como objetivo desenvolver e aplicar a técnica de PIXE diferencial para sondar a composição interna de fósseis da Bacia do Araripe de forma totalmente não destrutiva, superando as limitações das metodologias tradicionais que exigem corte ou polimento das amostras. Essa inovação não apenas preserva o patrimônio paleontológico, mas também abre novas perspectivas para a compreensão da biologia de organismos extintos e dos ambientes em que viveram em consonância com sua preservação. 

	\section{Fundamentação teórica}
\subsection{Descrição dos Fenômenos atrelados ao PIXE}
blablabla ionização
pegar base na dissertação do pedro, livre doc do manfredo e nesse aburaya 2005

\subsection{Equação fundamental do PIXE}
Iniciando com uma análise quantitativa do PIXE, como primeiro passo é necessário definir a forma de estabelecer cálculo de concentrações das substâncias analisadas, a partir dos dados de contagens dos raios X emitidos e das especificações do aparato experimental. O desenvolvimento teórico que se segue, baseou-se em \cite{Johansson_Campbell_1995,  Tabacniks_2005_LivreDocencia, Aburaya_2005, Campos_2010, Seabra_TCC, smitholc2004, SMIT20082047}

Introduzindo o conceito de seção de choque $\sigma$, como sendo uma medida da probabilidade de um evento ocorrer, e sendo definida experimentalmente no referencial do laboratório, sua equação diferencial por um angulo sólido $\Omega$ do detector é dada por $\frac{d\sigma}{d\Omega} = \frac{N_D} {N_P}\frac{A}{N_A} \frac{1}{\Omega}$, com $N_D$ sendo o número de fótons detectados, $N_P$ número de prótons incidentes $A$ a área frontal do alvo, e $N_A$ o número de núcleos do alvo \cite{Leo_1994_TechniquesNuclearParticle}.Para considerar que o número de fótons produzidos $dN_x$ é o mesmo do número detectado é necessário que não há interações físicas entre o fóton produzido e o detector, como em casos reais isso não se faz realidade, é necessário uma conversão destes dada a partir da eficiência $\epsilon$ do detector e de um fator de transmissão que considera a atenuação dos raio X até a superfície, dada pela lei de Beer-Lambert \cite{Mayerhofer_Pahlow_Popp_2023}, o número de fótons que irá emergir à superfície, irá ser atenuado por $e^{-\micro * x_{out}}$, com $\micro$ sendo o coeficiente de atenuação e $x_{out}$ o caminho de saída do fóton irradiado na amostra assim como representado pela figura \ref{fig:testelatex}.

\begin{figure}[H]
	\centering
	\includegraphics[width=0.6\linewidth]{../TesteLatex/bitmap.png}
	\caption{Representação gráfica geométrica do modelo de ionização simplificado.}
	\label{fig:testelatex}
\end{figure}


Desta forma o caminho de entrada percorrido pela próton, perdendo energia pelo poder de freamento é 
\begin{equation}
	x_{in}(E) = \int_{E_{0}}^{E_{f}} \frac{dE}{\rho S_M(E)}
	\label{eq:x_in} 
\end{equation}
com $\rho$ a densidade total; e sendo assim, pela geometria do problema o caminho de saída fica resultado como 
\begin{equation}
	x_{out}(E) = \frac{\cos(\alpha)}{\cos(\beta)}\int_{E_{0}}^{E_{f}} \frac{dE}{\rho S_M(E)}
	\label{eq:x_out} 
\end{equation}
Pela lei de Bragg \cite{Tabacniks_2005_LivreDocencia}, o poder de freamento da matriz M ($S_M(E)$) é a soma dos poderes de freamento individuais ($S_Z(E)$), ponderada pela concentração dos elementos presente, e é dada por $S_M(E) = - \rho ^{-1} \frac{dE}{dx}$. Definindo somente um fator de transmissão em função de um coeficiente de atenuação de massa $\micro / \rho$ como 
\begin{equation}
	T_z(E) = exp \big[-\big(\frac{\micro}{\rho}\big)_{M,Z} \frac{\cos(\alpha)}{\cos(\beta)}\int_{E_{0}}^{E_{f}} \frac{dE}{S_M(E)}\big]
	\label{eq:T(E)}
\end{equation}

E pela lei de Bragg também, o coeficiente de atenuação  $\micro$ será combinação linear dos componentes
da matriz de modo $\mu_i = \sum_{n} \frac{\rho_n}{\rho} \, \mu_{i,n}$ com $\rho_n$ a concentração do elemento componente da matriz \cite{Aburaya_2005}.

Desta forma, para o número de partículas detectadas, teremos $N'_{D} = N_x T_Z(E) \epsilon$, de forma que $N_X$ se iguala ao valor antigo de $N_D$. 

Analisando somente uma fração do elemento, dado por uma variável $dx$ de profundidade, teremos uma fração dos raios X em função desta profundidade e integrando o angulo sólido analisado do detector, teremos o numero de fótons detectados como  

\begin{equation}
	dN'_{Dx} = T_Z(E) \epsilon {N_P} \frac{N_A}{A} \frac{\Omega}{4 \pi} \sigma_x dx
	\label{eq:dN'_X}
\end{equation}
com agora $\sigma_x$ explicito como a seção de choque de produção de raio x, e  $dx$ sendo a profundidade da amostra. Convertendo a variável para energia por \ref{eq:x_in}, separando a  analise para apenas um elemento $Z$ com uma transição $i$, atenuando pela lei de Beer-Lambert, e considerando um detector real com uma eficiência $\epsilon_Z^i$, referente a este elemento e esta transição, resulta-se em um número de fótons detectados como

\begin{equation}
	dN_{Di}^Z = \epsilon_Z^i(\frac{\Omega}{4 \pi})  \frac{ N_P N_A^Z \sigma(E) T_Z(E)}{A \rho S_M(E)}dE
	\label{eq:dY_i^Z}
\end{equation}

Para mapear todo o número de fótons detectados ao longo da profundidade da amostra basta integrar para toda a energia, considerando uma inicial $E_0$ e a final $E$. Convertendo o número de próton para $\frac{Q}{e}$ já que o valor medido pelo equipamento é a corrente do material, e também $\frac{N_A^Z }{A} =\frac{m_A^Z N_0}{A M_A^Z}$ passando de $at/cm^2$ para $g/cm^2$ já que $M_A^Z$ é a massa atômica de Z, $m_A$ a massa da região e $N_0$ o número de Avogrado


\begin{equation}
	Y_{i}^Z = \frac{Q}{e} \frac{N_0}{M_A^Z} \frac{m_A^Z}{A \rho } \epsilon_Z^i(\frac{\Omega}{4 \pi}) \int_{E_{0}}^{E_{f}} \frac{\sigma(E) T_Z(E)}{S_M(E)}dE
	\label{eq:Y^Z}
\end{equation}
e pode-se notar que o termo $\frac{m_A}{A \rho }$ se resume a uma concentração do elemento Z, podendo portanto ser substituída por $C_Z$.

Por último é de extrema importância que o termo da seção de choque seja corretamente analisado, dado que o que estamos observando até o momento é a seção de choque de produção de raios X detectada pelo aparato. Porém como estamos trabalhando com ionização por protons, a seção de choque de ionização deste aparato difere por um fator de rendimento de fluorescência $\omega_Z$, e outro da fração da intensidade da linha $b_Z$ de modo que resultaremos na equação fundamental para o PIXE

\begin{equation}
	Y_{i}^Z = \frac{Q}{e} \frac{N_0 \omega_Z b_Z }{M_A^Z} C_Z \epsilon_Z^i(\frac{\Omega}{4 \pi}) \int_{E_{0}}^{E_{f}} \frac{\sigma_I(E) T_Z(E)}{S_M(E)}dE
	\label{eq:FUNDPIXE}
\end{equation}


\subsection{Aproximação de alvos finos}
Quando o elemento analisado tem uma espessura fina o suficiente, para considerar que não há absorção dos fótons produzidos pelo próprio corpo, ou seja $T_Z(E) \approx 1$, desenvolvido a partir de \cite{Tabacniks_2005_LivreDocencia, Tabacniks_AnalisePIXE_RBS, Campos_2010}.

Trabalhando a partir da equação \ref{eq:FUNDPIXE}, tem-se para essa aproximação

\begin{equation}
	Y_{i}^Z = \frac{Q}{e} \frac{N_0 \omega_Z b_Z }{M_A^Z} C_Z \epsilon_Z^i(\frac{\Omega}{4 \pi}) \sigma_I(E_0)
	\label{eq:alvfin1}
\end{equation}
onde o termo integral simplifica-se apenas para o cálculo da seção de choque de ionização referente a energia inicial.

Nomeando um termo de calibração de alvos finos como $r_i$, a equação se simplifica em uma relação linear da contagem de raio x com a concentração da substância tal como 

\begin{equation}
	Y_{i}^Z = {Q} r_i C_Z
	\label{eq:alvosfinos}
\end{equation}

desta forma é possível traçar relações simplificadas conhecido o parâmetro $r_i$, ou ainda o oposto, conhecendo uma concentração de uma amostra é possível identificar o parâmetro a partir disso.

\subsection{Formalismo para alvos grossos}

Para alvos grosso, teremos novamente o modelo complexo como descrito pela equação \ref{eq:FUNDPIXE}, porém por facilidade experimental, é possível estender a equação \ref{eq:alvosfinos} para que mantenha a constante de calibração, tendo os embasamentos desenvolvidos por \cite{Tabacniks_2005_LivreDocencia, Campbell_Cookson_Paul_1983, T_Anal}.

Iniciando unindo as equações \ref{eq:FUNDPIXE} e \ref{eq:alvosfinos}, e normalizando a seção de choque variável, pela seção de choque na energia inicial.

\begin{equation}
	Y_{i}^Z = {Q} r_i C_Z \int_{E_{0}}^{E_{f}} \frac{\sigma_I(E)}{\sigma_I(E_0)} \frac{T_Z(E)}{S_M(E)}dE
	\label{eq:Alvos_grossos_inic}
\end{equation}

E nomeando uma variável como $R_M$ em que seu valor seja o da integral da equação anterior. A equação se simplifica portanto de uma forma linear, como

\begin{equation}
	Y_{i}^Z = {Q} R_M r_i C_Z 
	\label{eq:Alvos_grossosf_fin}
\end{equation}


\subsection{Modelo semiempírico de Johansson e Campbell da seção de choque de ionização}

blablabla pegar livre doc do manfredo. Se não rolar, voltar pro isics

\subsection{Atribuição do coeficiente de atenuação e do poder de freamento}

pegar livre doc do manfredo, se não rolar pegar refs no vídeo do tiago

\subsection{Aplicação do PIXE Diferencial}

Recalcular as coisas pra esse modelo...

Para a realização da análise, será utilizado PIXE diferencial. Quando se deseja determinar concentrações de elementos químicos em função da profundidade, são realizadas medidas sequenciais com variação progressiva de energia; a resolução da análise depende do valor do incremento e do número de incrementos de energia \cite{SmitHolc2004}. 

Algoritmos de deconvolução podem ser aplicados para caracterizar a composição química em profundidades específicas \cite{SmitHolc2004, SMIT20082047}. O algorítimo se baseia em resolver a equação do rendimento de raio-X $Y_{i}^{k} = \sum_{j=1}^{N} Y_{ij}^{k}$, onde i é referente ao elemento, k o número da medida, j as camadas e N o número máximo de medidas. O que faremos é considerar estrutura composta, como sendo divididas em sub-regiões (j) homogêneas, e então será analisada individualmente cada região, de modo


\begin{equation}
	Y_{ij}^{k} = \frac{\Delta \Omega}{4\pi} N_p^{k} N_A 
	\frac{\varepsilon_i \eta_i^{k}}{M_i} x_{ij} P_{ij}
	\int_{E_{j+1}^{k}}^{E_{j}^{k}} 
	\sigma_i^{x}(E) \exp(-\mu_{ij} \xi) 
	\frac{\mathrm{d}E}{S_j(E)}.
	\label{eq:yields1}
\end{equation}

Definindo $A_{k} \equiv \frac{\Delta \Omega}{4\pi} N_{p}^{k} N_{A}$ como número de próton generalizado,  

$T_{ij}^k \equiv P_{ij} \int_{E_{j+1}^{k}}^{E_{j}^{k}} \sigma_i^{x}(E) \exp(-\mu_{ij} \xi)\frac{\mathrm{d}E}{S_j(E)}$ como \textit{Thick Target Factor} e $Y_{i}^{k*} = Y_{i}^{k} \frac{M_{i}}{\epsilon_{i} \eta_{i}^{k}}$ como o rendimento generalizado, teremos uma equação de matriz
\begin{equation}
	Y_{i}^{k*} = A_{k} \sum_{j} T_{ij}^{k} x_{ij}.
	\label{eq:Yields_reduc}
\end{equation}
onde $x_{ij}$ é a concentração do elemento i na camada j. Somente podemos discretizar essa concentração da integral aproximando a concentração à uma função degrau, em relação as camadas, sendo homogênea perante a divisão realizada. O elemento de matriz $T_{ij}^k$ depende da integral de (\ref{eq:yields1}) e de $P_{ij} = \exp\left( -\sum_{l<j} \mu_{il} d_l / \cos\psi \right)$, onde $d_l$ e $\mu_{ij}$ são a grossura e o coeficiente de atenuação respectivamente. Como o \textit{thickness target factor} tem contido o $x_{ij}$, é necessário um procedimento iterativo para determinar ambos valores. $A_k$ é definido com base no aparato e modelo experimental.

	
	\section{Objetivos} 
O relatório tem por objetivo demonstrar o desenvolvimento e a aplicação de uma nova forma de análise composicional de fósseis por meio da física, visando a maior compreensão dos processos de fossilização.

O processo se baseia em realizar simulações com PIXE diferencial em padrões análogos a fósseis da Formação Crato, determinando os valores de energia necessários para sondar a composição química em diferentes profundidades.

	\section{Materiais e Métodos}
\subsection{Materiais de estudo} 

Para este estudo, foram utilizados tecidos musculares da vértebra de um peixe -- amostra GP2E7781G -- e tecidos do proventrículo(órgão do sistema digestório) preservado de um grilo -- amostra GP1E11237C. Os fósseis estudados são provenientes da Coleção Científica de Paleontologia do Instituto de Geociências da Universidade de São Paulo.

\subsection{Métodos de preparação}
\subsubsection{Estudos iniciais}
Inicialmente, foi feito um estudo acerca das composições dos fósseis estudados, identificando a espessura e os elementos químicos associados a diferentes camadas das amostras, obtidos a partir da análise por FIB-SEM-EDS, a qual foi realizada previamente no Laboratório Nacional de Nanotecnologia (LNNano), pertencente ao Centro Nacional de Pesquisa em Energia e Materiais (CNPEM). Essa etapa foi apresentada no primeiro relatório, tendo sido possível utilizar as medições para a produção de padrões com espessuras similares às encontradas nos tecidos preservados e realizar simulações para a construção de um modelo que será replicado nos fósseis. Para o caso da preservação do espécime GP1E11237C teremos $\ce{CaHPO_4}$ em cima de óxidos e hidróxidos de ferro, e para piritização originalmente consideramos $Fe_2O_3$, que representa a oxidação de pirita, sobre fosfatos de cálcio.

\subsubsection{Produção de pastilhas}
Nesta etapa do trabalho, foram realizadas simulações dos tecidos, em relação à distribuição de cada composto por sua espessura, no formato de pastilhas duplas. Iniciou-se pela confecção de pastilhas com padrões comerciais de $\ce{CaHPO4}$ e $\ce{Fe2O3}$, ambos da marca Dinâmica com 95\% de pureza. Para atestar os compostos presentes neste materiais, realizou-se espectroscopia de fluorescência de raios X, no sistema constituído por fonte de raios X mini Amptek e detector SDD Amptek XR-100 FAST SDD, com 25V e 300 segundos de aquisição, um método de análise por emissão atômica utilizado para caracterização elementar \cite{etde_7035934}. As medidas foram realizadas no Laboratório de Arqueometria e Ciências Aplicadas ao Patrimônio Cultural (LACAPC), no IFUSP. Neste processo, foi somente identificada uma contaminação de $Fe$ no padrão de $CaHPO_4$, que foi levada em consideração no processo de análise.

\begin{figure}[H]
	\begin{minipage}[b]{0.60\linewidth}
		\setlength{\parindent}{2em} % ativa a indentação
		Seguiu-se com a produção das pastilhas em uma prensa hidráulica manual — marca Beckman Fig. (\ref{fig:prensa_hidraulica_manual}), com capacidade máxima de 25 toneladas empregada para a compactação do material particulado. O procedimento consiste em posicionar o material dentro do pastilhador, ajustando-se a quantidade em função do resultado desejado, e com cautela para garantir uma compactação homogênea. Em seguida, aplica-se gradualmente a pressão até atingir o valor desejado. 
		
		Também foi utilizado um pastilhador de 5 mm de diâmetro, em conjunto com uma base e seu êmbolo conforme ilustrado na Figura \ref{fig:pastilhador}, empregado para moldar e comprimir o pó sob pressão controlada. Os pastilhadores são de propriedade e confecção própria do Laboratório Aberto de Física Nuclear (LAFN) do IF-USP, feitos em aço inoxidável.
	\end{minipage}
	\hfill
	%--- Coluna das figuras à direita ---
	\begin{minipage}[b]{0.36\linewidth}
		\centering
		\begin{subfigure}[b]{\linewidth}
			\centering
			\caption{ }\includegraphics[width=0.78\linewidth]{Métodos/prensa_hidraulica_manual-1.png}
			\label{fig:prensa_hidraulica_manual}
		\end{subfigure}
		
		
		\begin{subfigure}[b]{\linewidth}
			\centering
			\includegraphics[width=0.79\linewidth]{Métodos/pastilhador-1.png}
			\caption{Figura 1 (a): Prensa 
				hidráulica. \\ (b):Pastilhador}
			\label{fig:pastilhador}
		\end{subfigure}
	\end{minipage}
	\label{fig:equipamentos_pastilhas}
\end{figure}

Foram realizados dois modelos de pastilhas: o primeiro simulando o processo de piritização exterior, ou seja, com uma camada fina do padrão de $Fe_2O_3$ sobreposta à camada grossa de $CaHPO_4$ 2(a); o segundo segue o mesmo processo, porém invertendo-se os compostos 2(b). 

\begin{figure}[H]
	\centering
	\begin{subfigure}[b]{0.45\linewidth}
		\centering
		\includegraphics[width=0.85\linewidth]{Métodos/path2625-1.png}
		\caption{a) Modelo 1: Patilha de $Fe_2O_3$ sobre $CaHPO_4$.}
		\label{fig:filmeFe2O3_CaO}
	\end{subfigure}
	\hfill
	\begin{subfigure}[b]{0.45\linewidth}
		\centering
		\includegraphics[width=0.85\linewidth]{Métodos/path2624-1.png}
		\caption{b) Modelo 2: Patilha de $CaHPO_4$ sobre $Fe_2O_3$.}
		\label{fig:filmeCaO_Fe2O3}
	\end{subfigure}
	\caption{Modelos de produção de filmes finos em diferentes combinações de substrato e material depositado.}
	\label{fig:modelos_filmes}
\end{figure}

A base da pastilha foi prensada  inicialmente com uma carga de 1 tonelada por 10 segundos para formar uma estrutura uniforme. Sobre esta base, o padrão superior foi depositado, e uma segunda prensagem foi aplicada, desta vez com 3 toneladas por 30 segundos. As quantidades de material utilizados foram estimadas pensando em um corpo verde ideal, com densidade relativa máxima, logo, buscando obter a espessura estimada na primeira etapa deste trabalho - entre $0.9\mu m$ e $40 \mu m$ para o $Fe_2O_3$, e entre $0.2\mu m$ e $0.7 \mu m$ para o $CaHPO_4$.

Para o modelo 1, foram realizadas três pastilhas, utilizando aproximadamente 50 mg de $CaPO_4$ na base e variando-se entre 4 e 11 mg de $Fe_2O_3$ \ref{fig:pellet1}, tendo uma boa estabilidade após a prensagem e podendo-se estimar uma espessura da camada superior próxima a $50 \mu m$. Nota-se que essas espessuras correspondem apenas à camada superficial, mais externa (Fig. 2), visto que o interesse do experimento está na sondagem da interface entre compostos distintos. Como as dimensões extrapolam às desejadas, máximo de 40 um, foi realizado um novo processo para tentar suprir esta adversidade, prensando duas pastilhas com a mesma quantidade de $CaHPO_4$, porém agora o padrão de $Fe_2O_3$ foi adsorvido na pastilha já confeccionada, a partir de uma solução do mesmo com álcool isopropílico Fig. \ref{fig:pellet2}. A espessura das últimas duas não pôde ser estimada pelo alto número de incertezas que envolvem o processo; porém, notou-se que a distribuição deste composto na pastilha foi difusa, de modo a ter uma concentração não heterogênea em contrapartida às primeiras, e ter menos de 3 mg, visto que essa quantidade foi utilizada para realizar a solução e ainda restou material após a adsorção. Para este modelo 1, os três primeiros padrões foram nomeados como "pastilhas duplas", e os dois últimos como "pastilhas adsorvidas".

\begin{figure}[H]
	\centering
	\begin{subfigure}[b]{0.3\linewidth}
		\centering
		\includegraphics[width=0.8\linewidth]{Métodos/pellet1-1.png}
		\caption{a) Pastilha dupla.}
		\label{fig:pellet1}
	\end{subfigure}
	\hfill
	\begin{subfigure}[b]{0.3\linewidth}
		\centering
		\includegraphics[width=0.8\linewidth]{Métodos/pellet2-1.png}
		\caption{b) Pastilha adsorvida.}
		\label{fig:pellet2}
	\end{subfigure}
	\hfill
	\begin{subfigure}[b]{0.3\linewidth}
		\centering
		\includegraphics[width=0.77\linewidth]{Métodos/pellet3-1.png}
		\caption{c) Pastilha tripla.}
		\label{fig:pellet3}
	\end{subfigure}
	
	\caption{Pastilhas produzidas a depender dos diferentes modelos.}
	\label{fig:pelletfinal}
\end{figure}

Já no caso das pastilhas do modelo 2, foram realizadas também 3 pastilhas. As limitações que perduraram esta produção envolveram essencialmente a baixa adesão e alta porosidade do óxido de ferro. No processo de desenformar-las, suas base se rompiam por estar muito quebradiças, necessitando de uma mudança de planejamento. Foi pensado em fazer uma "pastilha tripla" Fig. \ref{fig:pellet3}, sendo a base de um material que não afetariam as análises posteriores, como o ácido bórico, seguido da base inicial que seria o óxido, e por último o fosfato bibásico. A pastilha tripla, assim nomeada, teve uma estruturação melhor em relação à somente a base de $Fe_2O_3$, porém o mínimo de material de fosfato utilizado foi na ordem de 5 mg, adquirindo espessuras muito superiores a $0,7\mu m$ (o limite máximo de espessura).


Como foi evidenciado, no processo de preparação das pastilhas, as espessuras não atenderam, e, quando muito, apenas se aproximaram do valor ideal, necessitando uma forma mais refinada de produção para se adequar melhor tanto às limitações experimentais quanto ao próprio processo de fossilização. Para isto, iniciou-se uma etapa de produção de filmes finos dos elementos encontrados em conjunto com o Laboratório de Micro e Nano Fabricação do Instituto Mackenzie de Pesquisas em Grafeno e Nanotecnologias (Lab-MackNano), e com Laboratório de Sistemas Integráveis (LSI) da Escola Politécnica da USP (POLI-USP).

Como as pastilhas de modelo 1, com óxido de ferro superior, atingiram espessuras próximas, mesmo extrapolando o limite, foi escolhido iniciar o processo de confecção dos filmes finos pelo modelo 2. Todo este processo esta sendo desenvolvido até o momento no Lab-MackNano. 

O fosfato bibásico, em sua sinterização e, posteriormente, sua evaporação por \textit{sputtering}, tem um comportamento não trivial, pois ele se decompõem em temperaturas baixas, inicialmente próximo a 400ºC \cite{AcuneytCaHPO4}. Como alternativa, foi escolhido calcinar este material, a 1000ºC na atmosfera ambiente, por duas horas em uma mufla Quimis Q-318M24, presente no Laboratório de Dosimetria do IF-USP. Na calcinação, o fosfato de cálcio bibásico perde água, tanto a absorvida pela umidade do meio, quanto de sua própria estrutura pelo processo 

\[
\ce{2 CaHPO4.2H2O ->[ \Delta ] Ca2P2O7 + 5 H2O}
\]

formando o pirofosfato de cálcio $\ce{Ca2P2O7}$. A decomposição da forma dihidratada do fosfato de cálcio em anidro ocorre próximo a 180ºC, e a formação do pirofosfato se inicia próximo a 450ºC e perdura até temperaturas superiores\cite{AcuneytCaHPO4}. Após isso, ocorre a mudança de fase do pirofosfato, sendo formado o $\ce{\beta-Ca2P2O7}$ entre 850ºC a 1000ºC. A escolha da temperatura de calcinação ser 1000ºC foi motivada por marcar o fim da transição de fase do pirofosfato e de já não ter fosfato bibásico presente no composto. 

Tendo o material de pirofosfato sido produzido, foi analisado para identificar a mudança estrutural, por meio de espectroscopia Raman, um outro método de análise composicional. Esta técnica identifica as vibrações das ligações moleculares por meio do espalhamento do feixe de luz incidido na amostra. O equipamento utilizado (EZRaman-I Series High Performance Portable Raman Analyzer) empregando o laser de 785 nm, com 30 segundos de aquisição. Os resultados (\ref{fig:Raman}) apresentam uma diferença significativa, tendo perdido picos característicos das ligações do fosfato. Além disto, surgiram picos em valores mais altos de deslocamento Raman (cm-1), que podem representar possíveis ligações de carbono com nitrogênio, devido a uma contaminação, já que a produção foi realizada em atmosfera comum. Contudo, isso não afeta os resultados desejados.

\begin{figure}[H]
	\centering
	\includegraphics[width=\linewidth]{Métodos/plotest-1.png}
	\caption{Gráfico da intensidade em unidades arbitrárias pelo \textit{Raman shift} das amostras antes e depois da calcinação}
	\label{fig:Raman}
\end{figure}



A sinterização do pirofosfato já é conhecida e tem melhores resultados em comparação ao fosfato de cálcio bibásico \cite{Wang1993}, visto que a calcinação deste não pode envolver temperaturas elevadas. Inicialmente, foi realizado um teste para identificar a densidade relativa do alvo sinterizado, afim de descobrir se seria possível a evaporação; assim, o pirofosfato foi moído e pastilhado inicialmente com $4.02 g$, utilizando um pastilhador de $20 mm$ de diâmetro, utilizando-se 8 toneladas. O corpo verde ficou com uma densidade relativa de aproximadamente 80\%. Também foi realizado uma pastilha de $\ce{CaHPO4}$ com as mesmas especificações do pirofosfato, porém com uma massa de $4,06g$, o que resultou em uma densidade relativa no corpo verde de aproximadamente 60\%.

\begin{figure}[H]
	\centering
	\begin{subfigure}[a]{0.45\linewidth}
		\centering
		\includegraphics[width=0.8\linewidth]{Métodos/Texto do seu parágrafo (1080 x 1080 px)(1)-1.png}
	\end{subfigure}
	\hfill
	\begin{subfigure}[b]{0.45\linewidth}
		\centering
		\includegraphics[width=0.85\linewidth]{Métodos/Texto do seu parágrafo (1080 x 1080 px)-1.png}
	\end{subfigure}
	\caption{Resultado da sinterização: a esquerda os corpos verdes, e a direita o resultado final. As pastilhas marcadas com 1 são formadas pelo pirofosfato, e as marcadas com 2 foram feitas inicialmente com o fosfato bibásico.}
	\label{fig:past}
\end{figure}

O processo de sinterização foi realizado em atmosfera controlada, com 1000ºC durante 5 horas, resultando em duas pastilhas resistentes com uma densidade relativa maior do que a inicial, tendo aproximadamente 70\% e 85\% para $\ce{CaHPO4}$ e $\ce{Ca2P2O7}$ respectivamente. A pastilha de fosfato bibásico apresentou buracos fundos em sua superfície, possivelmente devido a saída da água da molécula durante a sinterização. 

O último passo para a produção do filme fino agora é a sinterização do alvo no tamanho ideal, visto que o teste inicial apresentou resultados satisfatórios, seguido da evaporação do mesmo sobre substrato ou sobre folhas de ferro oxidadas.

Para o modelo 1, com o óxido de ferro superior, também foi necessário realizar filmes finos, já que a espessura das pastilhas não se fez idêntica ao desejado. Para este processo, a confecção dos alvos não é complexa como do material anterior, e pela espessura e por ser um material metálico, a evaporação por \textit{Electron-Beam} será mais adequada, sendo esta realizada no LSI-USP. Para simplificar o processo, a evaporação será realizada sobre pastilhas de fosfato, ao invés de realizar um filme duplo. As pastilhas foram produzidas com aproximadamente $150 mg$ e 1 cm de diâmetro, coladas em cima de suportes de aço inox com um diâmetro de abertura de 0.8 cm.


\begin{figure}[H]
	\centering
	\includegraphics[width=0.5\linewidth]{Métodos/alvos-1.png}
	\caption{Pastilhas base para evaporação dos filmes de ferro produzidos e colados em seus suporte.}
	\label{fig:placehold}
\end{figure}

O próximo passo para este modelo é a realização da evaporação do ferro sobre estas pastilhas, seguida da oxidação das mesmas.









	\subsection{Métodos de Medição}

As amostras foram analisadas por PIXE no Laboratório de Análise de Materiais por Feixes Iônicos (LAMFI/USP). As medições foram realizadas utilizando um feixe de prótons com energia de \SI{2.6}{\mega\electronvolt} a \SI{3.4}{\mega\electronvolt} e corrente de \SI{3}{\nano\ampere} a \SI{5}{\nano\ampere}. O diâmetro do feixe foi ajustado para aproximadamente \SI{1}{\milli\metre}, e o tempo de irradiação de cada ponto foi de cerca de \SI{600}{\second}, garantindo boa estatística sem degradação perceptível das amostras.

O arranjo experimental se baseia no feixe externo, como se mostra como na figura \ref{fig:medicao_pixie_dupla}, onde a amostra esta posicionada a frente do feixe e do detector.
\begin{figure}[H]
	\centering
	
	\includegraphics[width=0.7\textwidth]{Métodos/Captura de tela de 2025-10-24 19-48-51-1.png}\\[1em]
	\includegraphics[width=0.7\textwidth]{Métodos/Captura de tela de 2025-10-24 19-49-27-1.png}
	\caption{Vistas do sistema experimental durante as medições PIXE realizadas no LAMFI: superior exibe uma visão geral do arranjo experimental; inferior exibe amostra de frente.}
	\label{fig:medicao_pixie_dupla}
\end{figure}


Os espectros obtidos foram posteriormente processados e calibrados em energia e intensidade, pelo software QXAS (\textit{Quantitative X-ray Analysis System}), permitindo a determinação quantitativa dos elementos presentes nas amostras. 


\subsection{Métodos de Análise}

 A solução envolve minimizar $\chi^2$ dado por 
\begin{equation}
	\chi^{2} = \sum_{i k} \left( \frac{Y_{i}^{k*}}{A_{k}} - \sum_{j} T_{ij}^{k} x_{ij} \right)^{2}.
\end{equation}
até encontrar valores satisfatórios para $x_{ij}$





	\section{Resultados}
\subsection{Alvos finos preliminares}

\begin{table}[H]
	\centering
	\caption{Valores de contagem de $Y^{Fe}_{K_\alpha}$ com incerteza, carga acumulada e espessura calculada para cada amostra.}
	\label{tab:dados_fe_incerteza}
	\begin{tabular}{c c c c}
		\hline
		Amostra & $Y^{Fe}_{K_\alpha}$(Contagens) & $Q$($\mu$C) & $d$($\mu$m) \\
		\hline
		1  & $1751(13)\times 10^{1}$ & 0.555 & 0.513259 \\
		2  & $841(92)\times 10^{1}$  & 0.534 & 0.256102 \\
		3  & $2381(16)\times 10^{1}$ & 0.484 & 0.800411 \\
		4  & $1140(11)\times 10^{1}$ & 0.494 & 0.375581 \\
		5  & $927(97)\times 10^{1}$  & 0.508 & 0.296902 \\
		6  & $2519(16)\times 10^{1}$ & 0.471 & 0.870076 \\
		7  & $2529(16)\times 10^{1}$ & 0.448 & 0.918701 \\
		8  & $4301(21)\times 10^{1}$ & 0.482 & 1.4519   \\
		9  & $2498(16)\times 10^{1}$ & 0.436 & 0.932073 \\
		10 & $5601(24)\times 10^{1}$ & 0.405 & 2.25031  \\
		\hline
	\end{tabular}
\end{table}



{\color{red} CALCULAR INCERTEZA, CONSIDERAR A DISTRIBUIÇÃO DE CARGA, E DO CALCULO DE SECCHOQUE}

\begin{figure}[H]
	\centering
	\begin{subfigure}[b]{0.495\linewidth}
	\includegraphics[width=\linewidth]{results/grafico 2 com pico2.png}
	\label{fig:graf2-fit}
	\end{subfigure}
	\hfill
	\begin{subfigure}[b]{0.495\linewidth}
	\includegraphics[width=\linewidth]{results/grafico 8 com pico.png}
	\label{fig:graf8-fit}
	\end{subfigure}
	\hfill
	\caption{Gráfico de contagens pelo canal com ajuste para a pastilha 2 (a) com $\chi^2 = 1.9$, e 8 (b) com $\chi^2 = 1.4$}
	
\end{figure}

\

\subsection{Pixe diferencial}
Foram realizadas medidas de PIXE com três valores de energia diferentes, nas pastilhas duplas e adsorvidas, para criar um padrão comparativo quando os filmes finos estivessem prontos. Com isso, as energias dos prótons incidentes foram de 2.6, 3.0 e 3.4 MeV com um tempo de 300 segundos de medição (Figs. \ref{figbipellet} e \ref{fig:ads} ).
\begin{figure}[H]
	\centering
	\includegraphics[width=\linewidth]{results/BLP-1-1.png}
	\caption{Gráfico de contagens pelo canal, das análises da pastilha duplas}
	\label{figbipellet}
\end{figure}
Observando os resultados obtidos para a pastilha dupla, é notável que a espessura superior da pastilha foi grossa o suficiente para impedir que os Raios-X emitidos transpassassem por ela, sendo absorvidos no trajeto. Foi possível identificar o Fe presente e uma baixa concentração de Ca que não variou com o aumento na energia.

\begin{figure}[H]
	\centering
	\includegraphics[width=\linewidth]{results/ADS-1-1.png}
	\caption{Gráfico de contagens pelo canal, das análises da pastilha adsorvidas}
	\label{fig:ads}
\end{figure}
Para esta última, os valores de $\chi^2$ para a pastilha dupla e adsorvida foram aproximadamente 5 e 3, respectivamente. Sendo assim, é notável a variação das contagens de Ca e P em função da energia, representando que com uma variação da energia, a variação das contagens mudou significantemente, mesmo normalizando pela corrente. Essa variação pode ser apenas referente ao raio X emitido pela energia, ou pode indicar que o feixe esta lendo camadas distintas da amostra. Esta questão será sanada no próximo passo, com a aplicação do algorítimo de deconvolução. 

	
\section{Atividades futuras }

Como demonstrado por este relatório, desafios quanto à produção dos padrões levantados no primeiro relatório foram superados. Porém, notou-se a necessidade de produção dos filmes finos, o que está em progresso em instituições parceiras (Poli-USP e Universidade Mackenzie). Para o relatório final, planeja-se terminar os padrões com filmes finos, analisar essas amostras com PIXE e realizar as simulações com o algoritmo.
	\section{Anexo}
Contagens iniciais das patilhas 2 e 8 respectivamente com ajuste de pico especificado por elemento 

\begin{figure}[H]
	\centering
	\includegraphics[width=\linewidth]{results/grafico 8 com picos sep.png}
	\caption{Gráfico de contagens pela energia com ajuste para a pastilha 8}
	\label{fig:graf8-fit/e}
\end{figure}
\begin{figure}[H]
	\centering
	\includegraphics[width=\linewidth]{results/grafico 2 com pico.png}
	\caption{Gráfico de contagens pela energia com ajuste para a pastilha 2}
	\label{fig:graf2-fit/e}
\end{figure}
	
	
	\addcontentsline{toc}{section}{Referências}
	\bibliographystyle{apalike}
	\bibliography{8mybib}
\end{document}
